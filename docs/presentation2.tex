\documentclass[11pt, sans]{beamer}
\usepackage[utf8]{inputenc}
\usepackage[export]{adjustbox}
\usepackage{graphicx}
\usepackage{pgfpages}
\usepackage{listings}
\usepackage{color}

\usetheme{JuanLesPins}

\newcommand{\R}{\mathbb{R}} 
\newcommand{\X}{\mathbf{X}}
\newcommand{\Xr}{\mathbf{X}_{train}} 
\newcommand{\Xt}{\mathbf{X}_{test}}
\newcommand{\y}{\mathbf{y}} 
\newcommand{\yr}{\mathbf{y}_{train}} 
\newcommand{\yt}{\mathbf{y}_{test}}
\newcommand{\Un}{\mathbf{1}} 

\definecolor{lightpurple}{rgb}{0.8,0.8,1}
 
\lstset{
breaklines=true,
basicstyle=\footnotesize,
numbers=left,
stepnumber=1,
numbersep=5pt,
numberstyle=\small\color{black},
keywordstyle=\color{black},
commentstyle=\color{black},
stringstyle=\color{black},
frame=single,
tabsize=2,
backgroundcolor=\color{lightpurple}
}

\title[Présentation]{Time series recognition : Performances}
\subtitle[\ldots]{Forecsys}
\author{Constantin Philippenko}
\institute[MIPT]{Moscou}
\date{\today}

\addtobeamertemplate{footline}{\insertframenumber/\inserttotalframenumber}

\begin{document}

\begin{frame}
\titlepage
\end{frame}

\begin{frame}
	\frametitle{Goal of the project}
	
	\begin{alertblock}{Objectif}
	Recognize the activity of workers looking on their acceleration movements. That implies a time series recognition study.
	\end{alertblock}
	
	\begin{block}{Two part}
	\begin{itemize}
		\item Modelisation of the problem :
			\begin{enumerate}
				\item Data Preparation
				\item Data Segmentation
				\item Template Construction
			\end{enumerate}
		\item Model exploitation	
			\begin{enumerate}
				\item Time Series Comparaison
				\item Time Series Classification
				\item Time Series Recognition
			\end{enumerate}			 
	\end{itemize}
	
	\end{block}
	
\end{frame}

\begin{frame}
	\frametitle{Introduction}
	
	The goals :
	\begin{itemize}
		\item Check that the classification is not random
		\item Check that a step is recognized as a step
		\item Check that a movement different of a step is not recognized as a step
		\item Check that our modelisation is relevant for a multi-classes classification
		\item Check that all the features are relevant, and if not find the useless ones.
		\item Find the best classifier
		\item Make the recognition processus as fast as possible
	\end{itemize}
		
\end{frame}

\begin{frame}
	\frametitle{Introduction}
		The datasets :
	\begin{itemize}
		\item The Forecsys Data : a binary classification : walking or not walking.
		\item The USC Data : a multi-classes classification (11 classes)
	\end{itemize}
\end{frame}

\begin{frame}
\frametitle{Introduction}
	The classifiers:
	\begin{itemize}
		\item LinearSVM with a dual view (more samples than features)
		\item Naives Bayes Gaussian
		\item k-Nearest Neighbors : \textit{choose the optimal k}
		\item Descision Tree : \textit{choose the maximal lenght, the maximum numbers of features used in each node}
		\item Random Forest : \textit{choose the number of trees and the trees parameters}	
	\end{itemize}		
\end{frame}		
	
\begin{frame}
\frametitle{Introduction}
	The features:
	\begin{itemize}
		\item The spatial-shift parameters : $w_0$, $w_1$ and the spatial distance between the series and the template
		\item The temporel-shift parameters : $w_2$, $w_3$ and the DTW distance between the series and the template
		\item The fourier transformation parameters : $\Re(fft_0)$, $\Re(fft_1)$ and $\Im(fft_1)$
		\item The dispersion of the series values\footnote{Very useful to quickly discriminate the standings activities}
	\end{itemize}
\end{frame}

\begin{frame}
\frametitle{Confusion matrices construction}

Data : $ \X, \y \underset{\text{splitted in two parts}}{\Longrightarrow} \Xr, \y_{train}$ and $ \Xt, \y{test}$

\bigskip

Model fitting:

\begin{itemize}
	\item Fit($\Xr,\yr$)
	\begin{itemize}
		\item Auto-recognition : predict($\Xr,\yr$)
		\item Hold-Out recognition : predict($\Xt, \yt$)
	\end{itemize}
	\item $\text{Fit}_{\text{3-fold}}(\X,\y) \Longrightarrow $ $\underset{i \in \{1,2,3\}}{mean}(\text{predict}(\X_{fold_i}, \y_{fold_i}))$
\end{itemize}

$\hookrightarrow$ At all, three confusion matrices for each classifiers.

\end{frame}

\begin{frame}
\frametitle{Check that the classification is not random}
\framesubtitle{Binary Classification}

 \begin{overprint}
   
	
        \onslide<1> 
        \begin{figure}[H]
            \begin{minipage}[c]{.46\linewidth}
                  \includegraphics[scale=0.35,center]{../report_pictures/binary_classification/Auto-recognition_performance_for_LinearSVC.png}
            \end{minipage} \hfill
            \begin{minipage}[c]{.46\linewidth}
                \includegraphics[scale=0.35,center]{../report_pictures/binary_classification/Hold-Out_recognition_performance_for_LinearSVC.png}
            \end{minipage}
            \caption{Auto and Hold-Out Validation for LinearSVC}
            \label{CV_LinearSVC}
        \end{figure}   
   
        \onslide<2> 
        \begin{figure}[H]
            \includegraphics[scale=0.46,center]{../report_pictures/binary_classification/3-fold_validation_for_LinearSVC.png}
        \caption{3-Fold Validation for LinearSVC}
        \label{3Fold_LinearSVC}
        \end{figure} 
        
        \onslide<3> 
        \begin{figure}[H]
            \begin{minipage}[c]{.46\linewidth}
                  \includegraphics[scale=0.35,center]{../report_pictures/binary_classification/Auto-recognition_performance_for_GaussianN.png}
            \end{minipage} \hfill
            \begin{minipage}[c]{.46\linewidth}
                \includegraphics[scale=0.35,center]{../report_pictures/binary_classification/Hold-Out_recognition_performance_for_GaussianN.png}
            \end{minipage}
            \caption{Auto and Hold-Out Validation for GaussianN}
            \label{CV_GaussianN}
        \end{figure}   
   
        \onslide<4> 
        \begin{figure}[H]
            \includegraphics[scale=0.46,center]{../report_pictures/binary_classification/3-fold_validation_for_GaussianN.png}
        \caption{3-Fold Validation for GaussianN}
        \label{3Fold_GaussianN}
        \end{figure} 
        
        \onslide<5> 
        \begin{figure}[H]
            \begin{minipage}[c]{.46\linewidth}
                  \includegraphics[scale=0.35,center]{../report_pictures/binary_classification/Auto-recognition_performance_for_KNeighbor.png}
            \end{minipage} \hfill
            \begin{minipage}[c]{.46\linewidth}
                \includegraphics[scale=0.35,center]{../report_pictures/binary_classification/Hold-Out_recognition_performance_for_KNeighbor.png}
            \end{minipage}
            \caption{Auto and Hold-Out Validation for KNeighbor}
            \label{CV_KNeighbor}
        \end{figure}   
   
        \onslide<6> 
        \begin{figure}[H]
            \includegraphics[scale=0.46,center]{../report_pictures/binary_classification/3-fold_validation_for_KNeighbor.png}
        \caption{3-Fold Validation for KNeighbor}
        \label{3Fold_KNeighbor}
        \end{figure} 
        
        \onslide<7> 
        \begin{figure}[H]
            \begin{minipage}[c]{.46\linewidth}
                  \includegraphics[scale=0.35,center]{../report_pictures/binary_classification/Auto-recognition_performance_for_DecisionT.png}
            \end{minipage} \hfill
            \begin{minipage}[c]{.46\linewidth}
                \includegraphics[scale=0.35,center]{../report_pictures/binary_classification/Hold-Out_recognition_performance_for_DecisionT.png}
            \end{minipage}
            \caption{Auto and Hold-Out Validation for DecisionT}
            \label{CV_DecisionT}
        \end{figure}   
   
        \onslide<8> 
        \begin{figure}[H]
            \includegraphics[scale=0.46,center]{../report_pictures/binary_classification/3-fold_validation_for_DecisionT.png}
        \caption{3-Fold Validation for DecisionT}
        \label{3Fold_DecisionT}
        \end{figure} 
        
        \onslide<9> 
        \begin{figure}[H]
            \begin{minipage}[c]{.46\linewidth}
                  \includegraphics[scale=0.35,center]{../report_pictures/binary_classification/Auto-recognition_performance_for_RandomFor.png}
            \end{minipage} \hfill
            \begin{minipage}[c]{.46\linewidth}
                \includegraphics[scale=0.35,center]{../report_pictures/binary_classification/Hold-Out_recognition_performance_for_RandomFor.png}
            \end{minipage}
            \caption{Auto and Hold-Out Validation for RandomFor}
            \label{CV_RandomFor}
        \end{figure}   
   
        \onslide<10> 
        \begin{figure}[H]
            \includegraphics[scale=0.46,center]{../report_pictures/binary_classification/3-fold_validation_for_RandomFor.png}
        \caption{3-Fold Validation for RandomFor}
        \label{3Fold_RandomFor}
        \end{figure} 
        
        \onslide<11> 
        \begin{figure}[H]
            \begin{minipage}[c]{.46\linewidth}
                  \includegraphics[scale=0.35,center]{../report_pictures/binary_classification/Auto-recognition_performance_for_LogisticR.png}
            \end{minipage} \hfill
            \begin{minipage}[c]{.46\linewidth}
                \includegraphics[scale=0.35,center]{../report_pictures/binary_classification/Hold-Out_recognition_performance_for_LogisticR.png}
            \end{minipage}
            \caption{Auto and Hold-Out Validation for LogisticR}
            \label{CV_LogisticR}
        \end{figure}   
   
        \onslide<12> 
        \begin{figure}[H]
            \includegraphics[scale=0.46,center]{../report_pictures/binary_classification/3-fold_validation_for_LogisticR.png}
        \caption{3-Fold Validation for LogisticR}
        \label{3Fold_LogisticR}
        \end{figure} 
         
	
 \end{overprint} 

\end{frame}

\begin{frame}
\frametitle{Check that the classification is not random}
\framesubtitle{Multi-Classification}

 \begin{overprint}
   
	
        \onslide<1> 
        \begin{figure}[H]
            \begin{minipage}[c]{.46\linewidth}
                  \includegraphics[scale=0.35,center]{../report_pictures/multi_classification/Auto-recognition_performance_for_LinearSVC.png}
            \end{minipage} \hfill
            \begin{minipage}[c]{.46\linewidth}
                \includegraphics[scale=0.35,center]{../report_pictures/multi_classification/Hold-Out_recognition_performance_for_LinearSVC.png}
            \end{minipage}
            \caption{Auto and Hold-Out Validation for LinearSVC}
            \label{CV_LinearSVC}
        \end{figure}   
   
        \onslide<2> 
        \begin{figure}[H]
            \includegraphics[scale=0.46,center]{../report_pictures/multi_classification/3-fold_validation_for_LinearSVC.png}
        \caption{3-Fold Validation for LinearSVC}
        \label{3Fold_LinearSVC}
        \end{figure} 
        
        \onslide<3> 
        \begin{figure}[H]
            \begin{minipage}[c]{.46\linewidth}
                  \includegraphics[scale=0.35,center]{../report_pictures/multi_classification/Auto-recognition_performance_for_GaussianN.png}
            \end{minipage} \hfill
            \begin{minipage}[c]{.46\linewidth}
                \includegraphics[scale=0.35,center]{../report_pictures/multi_classification/Hold-Out_recognition_performance_for_GaussianN.png}
            \end{minipage}
            \caption{Auto and Hold-Out Validation for GaussianN}
            \label{CV_GaussianN}
        \end{figure}   
   
        \onslide<4> 
        \begin{figure}[H]
            \includegraphics[scale=0.46,center]{../report_pictures/multi_classification/3-fold_validation_for_GaussianN.png}
        \caption{3-Fold Validation for GaussianN}
        \label{3Fold_GaussianN}
        \end{figure} 
        
        \onslide<5> 
        \begin{figure}[H]
            \begin{minipage}[c]{.46\linewidth}
                  \includegraphics[scale=0.35,center]{../report_pictures/multi_classification/Auto-recognition_performance_for_KNeighbor.png}
            \end{minipage} \hfill
            \begin{minipage}[c]{.46\linewidth}
                \includegraphics[scale=0.35,center]{../report_pictures/multi_classification/Hold-Out_recognition_performance_for_KNeighbor.png}
            \end{minipage}
            \caption{Auto and Hold-Out Validation for KNeighbor}
            \label{CV_KNeighbor}
        \end{figure}   
   
        \onslide<6> 
        \begin{figure}[H]
            \includegraphics[scale=0.46,center]{../report_pictures/multi_classification/3-fold_validation_for_KNeighbor.png}
        \caption{3-Fold Validation for KNeighbor}
        \label{3Fold_KNeighbor}
        \end{figure} 
        
        \onslide<7> 
        \begin{figure}[H]
            \begin{minipage}[c]{.46\linewidth}
                  \includegraphics[scale=0.35,center]{../report_pictures/multi_classification/Auto-recognition_performance_for_DecisionT.png}
            \end{minipage} \hfill
            \begin{minipage}[c]{.46\linewidth}
                \includegraphics[scale=0.35,center]{../report_pictures/multi_classification/Hold-Out_recognition_performance_for_DecisionT.png}
            \end{minipage}
            \caption{Auto and Hold-Out Validation for DecisionT}
            \label{CV_DecisionT}
        \end{figure}   
   
        \onslide<8> 
        \begin{figure}[H]
            \includegraphics[scale=0.46,center]{../report_pictures/multi_classification/3-fold_validation_for_DecisionT.png}
        \caption{3-Fold Validation for DecisionT}
        \label{3Fold_DecisionT}
        \end{figure} 
        
        \onslide<9> 
        \begin{figure}[H]
            \begin{minipage}[c]{.46\linewidth}
                  \includegraphics[scale=0.35,center]{../report_pictures/multi_classification/Auto-recognition_performance_for_RandomFor.png}
            \end{minipage} \hfill
            \begin{minipage}[c]{.46\linewidth}
                \includegraphics[scale=0.35,center]{../report_pictures/multi_classification/Hold-Out_recognition_performance_for_RandomFor.png}
            \end{minipage}
            \caption{Auto and Hold-Out Validation for RandomFor}
            \label{CV_RandomFor}
        \end{figure}   
   
        \onslide<10> 
        \begin{figure}[H]
            \includegraphics[scale=0.46,center]{../report_pictures/multi_classification/3-fold_validation_for_RandomFor.png}
        \caption{3-Fold Validation for RandomFor}
        \label{3Fold_RandomFor}
        \end{figure} 
        
        \onslide<11> 
        \begin{figure}[H]
            \begin{minipage}[c]{.46\linewidth}
                  \includegraphics[scale=0.35,center]{../report_pictures/multi_classification/Auto-recognition_performance_for_LogisticR.png}
            \end{minipage} \hfill
            \begin{minipage}[c]{.46\linewidth}
                \includegraphics[scale=0.35,center]{../report_pictures/multi_classification/Hold-Out_recognition_performance_for_LogisticR.png}
            \end{minipage}
            \caption{Auto and Hold-Out Validation for LogisticR}
            \label{CV_LogisticR}
        \end{figure}   
   
        \onslide<12> 
        \begin{figure}[H]
            \includegraphics[scale=0.46,center]{../report_pictures/multi_classification/3-fold_validation_for_LogisticR.png}
        \caption{3-Fold Validation for LogisticR}
        \label{3Fold_LogisticR}
        \end{figure} 
         
	
 \end{overprint} 

\end{frame}

\begin{frame}
\frametitle{Check that the classification is not random}
\framesubtitle{Cluster observation on the confusion matrices}

\begin{alertblock}{There are clearly four clusters in the time series}
\begin{itemize}
	\item Walking : it's very hard to dissociate the left/right/forward movements as well as the stairs movements
	\item Running
	\item Jumping
	\item The standing activities like sleeping, standing and sitting 
\end{itemize}

\end{alertblock}
\begin{exampleblock}{Expectation}
For the binary classification : we have dissociated the five steps. Thus, we expect a hight error of the second kind during the recognition process.
\end{exampleblock}
\end{frame}
	
\begin{frame}
\frametitle{Check that the classification is not random}
\framesubtitle{Recap tables}

\begin{overprint}
	
            \onslide<1>
            The following tables summarize the 3-fold validation percentages of success recognition for each of activity and for each classifier.
    
            \medskip
            
            Clearly, the random forest give the best results : a hight score on the both classes.
            
            \medskip
            
            On the other hand, the k-nearest neighbors classifier shows an huge over-fitting and may should be avoid.
            
    \begin{table}[h!]
    \centering
    \resizebox{\textwidth}{!}{
    \begin{tabular}{|c|c|c|c|c|c|c|}
        \hline
        Activity & LinearSVC & GaussianN & KNeighbor & DecisionT & RandomFor & LogisticR  \\
        \hline
        Step (\%) & 92.6 & 93.8 & 92.9 & 89.7 & 92.5 & 89.7  \\
 		\hline 
 		Not Step (\%) & 82.1 & 79.8 & 64.3 & 81.5 & 89.9 & 80.3  \\
 		\hline 
 		
    \end{tabular}}
    \end{table}
    
            \onslide<2>
    \begin{table}[h!]
    \centering
    \resizebox{\textwidth}{!}{
    \begin{tabular}{|c|c|c|c|c|c|c|}
        \hline
        Activity & LinearSVC & GaussianN & KNeighbor & DecisionT & RandomFor & LogisticR  \\
        \hline
        WalkingForward (\%) & 83.6 & 88.3 & 74.6 & 61.4 & 88.1 & 77.9  \\
 		\hline 
 		WalkingLeft (\%) & 34.5 & 25.2 & 57.0 & 67.1 & 77.1 & 31.6  \\
 		\hline 
 		WalkingRight (\%) & 64.4 & 64.9 & 39.0 & 66.4 & 62.2 & 57.2  \\
 		\hline 
 		WalkingUpstairs (\%) & 68.3 & 78.7 & 59.0 & 52.6 & 62.9 & 67.1  \\
 		\hline 
 		WalkingDownstairs (\%) & 66.1 & 44.8 & 32.1 & 59.5 & 70.4 & 65.0  \\
 		\hline 
 		RunningForward (\%) & 95.8 & 96.1 & 64.9 & 100.0 & 100.0 & 96.1  \\
 		\hline 
 		JumpingUp (\%) & 91.7 & 92.7 & 58.0 & 94.4 & 93.5 & 90.1  \\
 		\hline 
 		Sitting (\%) & 70.5 & 51.7 & 66.4 & 68.4 & 67.8 & 76.5  \\
 		\hline 
 		Standing (\%) & 27.5 & 65.9 & 15.0 & 82.5 & 77.7 & 17.7  \\
 		\hline 
 		Sleeping (\%) & 26.1 & 35.0 & 4.0 & 49.2 & 47.3 & 25.5  \\
 		\hline 
 		
    \end{tabular}}
    \end{table}
    
            
            \medskip
            
            At the first glance, the results looks like to be very poor for the multi-classes classification with almost all 
            rates below 65\%. 
            
            \medskip
            
            However, if we consider once again the confusion matrix and the four cluster cited above, the rates skyrocket to almost 
            95\% for the SVM, Gaussian, Decision Tree and Random Forest classifiers.

             
\end{overprint}	        

\end{frame}

\begin{frame}[fragile]

\begin{overprint}

\onslide<1>
\begin{lstlisting}
FUNCTION : Continuous recognition of a step
INPUT : a time series, a template, a fitted time series classificateur
FOR EACH points of the time series:
	select a sub-segment of the time series starting at the current point and with a length 10 % longer than the template
	compute the fft and the dispersion of the selected segment
	normalize the selected segment
	compute the temporal shift parameters
	performs the temporal shift on the normalized segment
	compute the spatial shift parameters
	carry out the classification of the segment 
	IF the segment is recognized as a step :
		go directly to the last point of the time and continue the loop
	ELSE :
		perform the recognition of the next point
\end{lstlisting}

\onslide<2>

\begin{figure}[H]
	\includegraphics[scale=0.3,center]{../report_pictures/continuous_recognition/example_comparison_process.png}
	\caption{Continous recognition : the recognition of a segment}
	\label{example_comparison_process}
\end{figure}

\onslide<3>

\begin{exampleblock}{Notes}

The recognition algorithm is very fast for a time series constitued of only steps. Indeed, once a step is detected, the algorithm directly skip to the end of it. Thus, fewer calculation are done. Contrariwise, in a time series without any steps, calulation are done for every points which considerably slows down the process.

\end{exampleblock}

\end{overprint}

\end{frame}


\begin{frame}
\frametitle{The binary classification}

\begin{alertblock}{Goal}
Measure the error of the first and second kind and to minimize them.
\end{alertblock}

\begin{exampleblock}{Method}
\begin{itemize}
	\item Error of the first kind : we have manually constructed a time series with $198$ step in a row. We carry out the coutinuous algorithm of step detection on it, and count how many times a step has been recognized.
	\item Error of the second kind : we have manually constructed a time series without any step. We perform the coutinuous algorithm of step detection on it, and count how many time a step has been recognized while it should not !
\end{itemize}
\end{exampleblock}

\end{frame}

\begin{frame}
\frametitle{The binary classification}
\begin{alertblock}{Current limits of this tests}
\begin{itemize}
	\item For the first case : a same step could be recognized several time\footnote{This is due to the poor accuracy of $w_2$ and $w_3$ in some particular point}, this fact could significantly pump up the rates. That explains why for some classifier we get 110 \% of recognition while all the steps have not been detected.
	\item The second time series has been constructed with several activities : standing, jumping, walking left/right/backward but without forward movement. So one should expected to recognize no step. However, the walking movement being really near, one will probably detect a lot of step and thereby increase the error of second kind.
\end{itemize}
\end{alertblock}

\end{frame}
	
\begin{frame}
\frametitle{The binary classification}
\framesubtitle{Check that a step is recognized as a step}

\begin{figure}[H]
	\includegraphics[scale=0.46,center]{../report_pictures/continuous_recognition/recognition_of_step_by_RandomFor.png}
	\caption{Countinous recognition of a step time series by a Random Forest Classifier}
	\label{a_step_is_a_step}
\end{figure}

\end{frame}


\begin{frame}
\frametitle{The binary classification}
\framesubtitle{Check that a step is recognized as a step}

\begin{overprint}
\onslide<1>
\begin{figure}[H]
	\includegraphics[scale=0.46,center]{../report_pictures/continuous_recognition/recognition_of_not_step_by_RandomFor.png}
	\caption{Continous recognition of a not-step time series by a Random Forest Classifier}
	\label{a_not-step_is_a_step}
\end{figure}

\onslide<2>
\begin{figure}[H]
	\includegraphics[scale=0.46,center]{../report_pictures/continuous_recognition/recognition_of_not_step_by_KNeighbor.png}
	\caption{Continous recognition of a not-step time series by a k-nearest neighbors algorithm}
	\label{a_not-is_sometimes_a_step}
\end{figure}


\end{overprint}	 

\end{frame}

\begin{frame}
\frametitle{The binary classification}
\framesubtitle{Tables recap}

	
    \begin{table}[h!]
    \centering
    \resizebox{\textwidth}{!}{
    \begin{tabular}{|c|c|c|c|c|c|c|}
        \hline
        Activity & LinearSVC & GaussianN & KNeighbor & DecisionT & RandomFor & LogisticR  \\
        \hline
        Step (number) & 179.0 & 193.0 & 196.0 & 175.0 & 183.0 & 185.0  \\
 		\hline 
 		Not Step (number) & 148.0 & 195.0 & 246.0 & 116.0 & 87.0 & 160.0  \\
 		\hline 
 		
    \end{tabular}}
    \end{table}
    
    \begin{table}[h!]
    \centering
    \resizebox{\textwidth}{!}{
    \begin{tabular}{|c|c|c|c|c|c|c|}
        \hline
        Activity & LinearSVC & GaussianN & KNeighbor & DecisionT & RandomFor & LogisticR  \\
        \hline
        Step (\%) & 89.9 & 96.9 & 98.4 & 87.9 & 91.9 & 92.9  \\
 		\hline 
 		Not Step (\%) & 61.3 & 66.0 & 67.1 & 59.9 & 62.6 & 63.3  \\
 		\hline 
 		
    \end{tabular}}
    \end{table}
    

\end{frame}

\begin{frame}
	\frametitle{TODO : What I have to do}
	
	\begin{itemize}
	\item Formally measure the speed of the comparison
	\item Select the best parameters for the classification
	\item Carry out a LB\_keogh comparaison on the times series so as to fast up the comparaison process
	\item Found a time series interpolation algorithm\footnote{So as to properly compute $w_0$ and $w_1$}
	\item Fast up the recognition process
	\item Write all the test (incomplet at the present time)
	\item Write all the documentation (incomplet at the present time)
	\item Write an essay dealing with this topic
	\end{itemize}
\end{frame}	

%\begin{frame}[allowframebreaks]
%        \frametitle{References}
%        \bibliographystyle{bibidx/fr-plain}
%        \bibliography{biblio.bib}
%\end{frame}



\end{document}