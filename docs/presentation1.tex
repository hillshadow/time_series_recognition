\documentclass[11pt, sans, handout]{beamer}
\usepackage[utf8]{inputenc}
\usepackage[export]{adjustbox}
\usetheme{JuanLesPins}

\newcommand{\R}{\mathbb{R}} 
\newcommand{\Un}{\mathbb{1}} 

\title[Présentation]{Time series recognition}
\subtitle[\ldots]{Forecsys}
\author[Dupont]{Constantin Philippenko}
\institute[MIPT]{Moscou}
\date{\today}

\addtobeamertemplate{footline}{\insertframenumber/\inserttotalframenumber}

\begin{document}

\begin{frame}
\titlepage
\end{frame}

\begin{frame}
	\frametitle{Goal of the project}
	
	\begin{alertblock}{Objectif}
	Recognize the activity of workers looking on their acceleration movements. That implies a time series recognition study.
	\end{alertblock}
	
	\begin{block}{Two part}
	\begin{itemize}
		\item Modelisation of the problem :
			\begin{enumerate}
				\item Data Preparation
				\item Data Segmentation
				\item Template Construction
			\end{enumerate}
		\item Model exploitation	
			\begin{enumerate}
				\item Time Series Comparaison
				\item Time Series Classification
				\item Time Series Recognition
			\end{enumerate}			 
	\end{itemize}
	
	\end{block}
	
\end{frame}

\begin{frame}
	\frametitle{Data Preparation} 
	
	 \begin{alertblock}{Goal}

		Make data the most informative and make them easy to manipulate

	\end{alertblock}
	
	\begin{columns}
	
	\begin{column}{0.5\textwidth}
		\begin{block}{The dataset}

		\begin{itemize}
			\item 14 subjects
			\item 12 activities
			\item 5 trials for each subject
			\item acceleration (unity $g$)
			\item gyroscope (unity $dps$)
		\end{itemize}

		\end{block}
	\end{column}
	
	\begin{column}{0.5\textwidth}
		\begin{block}{The activities}
		
		\begin{itemize}
			\item WalkingForward, WalkingLeft, WalkingRight
			\item WalkingDownstairs, WalkingUpstairs
			\item RunningForward
			\item Jumping
			\item Sitting, Standing, Sleeping
			\item EscalatorUp, EscalatorDown
		\end{itemize}
		
		\end{block}
	\end{column}
	
	\end{columns}
	
\end{frame}

\begin{frame}
\frametitle{Data Preparation}

	We focus on the acceleration and compute the sum of square of the three axis components :
	
	\[ SSQ = \sum A_x^2 + A_y^2 + A_z^2 \]
	
	where $A_t$ is the acceleration corresponding to the t-axes.
	
	\begin{figure}[H]
		\centering
		\includegraphics[scale=0.15]{../data/Series/ElevatorUp.png}
		\caption{Serie example : Elevator Up}
		\label{ElevatorUp}
	\end{figure}
	
\end{frame}

\begin{frame}
	\frametitle{Time Series Segmentation}
	 
	 \begin{alertblock}{Goal}

		Find the elementary time series of each activities. The segmentation is needed to compute afterward the template of the classe. There is a strong correlation between the segmentation quality and the template quality, and so on the recognition quality.

	\end{alertblock}
	
	\begin{block}{Two ways to segment the time series}
	\begin{itemize}
		\item automatically
		\item manually
	\end{itemize}
	\end{block}

\end{frame}

\begin{frame}
	\frametitle{Time Series Segmentation}
	\framesubtitle{Automatically}
	
	The automatic method have been implemented using an idea developped by Christian Derquenne\cite{derquenne}. 
	
	This method is based on the smoothing and the differenciation of the time series, before carrying out a breaking points selection.	
	
	\begin{figure}
	\includegraphics[scale=0.4,center]{../USC-Activities/JumpingUp/automatic/breaking_points.png}
	\caption{Automatic segmentation of JumpingUp}
	\end{figure}

\end{frame}

\begin{frame}
	\frametitle{Time Series Segmentation}
	\framesubtitle{Automatically}
	
	\begin{figure}
	\includegraphics[scale=0.4,center]{../USC-Activities/Sleeping/automatic/breaking_points.png}
	\caption{Automatic segment of Sleeping}
	\end{figure}
	
	\begin{exampleblock}{Problem}
	How to segment Sleeping, Standing or Sitting ?
	\end{exampleblock}
	
\end{frame}

\begin{frame}
	\frametitle{Time Series Segmentation}
	\framesubtitle{Manually}
	
	If we consider more points and look more carefully, one could notice that the peak  are going two by two. As a consequence a human will choose a segmentation which incorporate two peaks and not only one in each segment. 
		
	\begin{figure}
      \includegraphics[scale=0.4,center]{../USC-Activities/WalkingForward/automatic/breaking_points.png}
      \caption{Automatic segmentation of WalkingForward}
    \end{figure}
\end{frame}

\begin{frame}
	\frametitle{Time Series Segmentation}
	\framesubtitle{Manually}
	
	\begin{figure} 	
      \includegraphics[scale=0.4,center]{../USC-Activities/WalkingForward/manual/breaking_points.png}
      \caption{Manually segmentation of WalkingForward}
	\end{figure}
	
	\begin{exampleblock}{Advantage}
	Improve the accuracy of the recognition, indeed we will got a longer template, which contains more information.
	\end{exampleblock}
	
\end{frame}

\begin{frame}
	\frametitle{Time Series Segmentation}
	\framesubtitle{Results}
	
 Each segments is vertically normalized. This normalization is performed so as to allow a comparaison between them.
 
 	\[ s_{normalized_i} = \frac{s_i - average(s)}{dispersion(s)} \]
		
	\begin{figure}
   	\begin{minipage}[c]{.46\linewidth}
      \includegraphics[scale=0.16,center]{../USC-Activities/WalkingForward/automatic/segments_superposition.png}
      \caption{Automatic segmentation of WalkingForward}
   	\end{minipage} \hfill
   	\begin{minipage}[c]{.46\linewidth}
      \includegraphics[scale=0.16,center]{../USC-Activities/WalkingForward/manual/segments_superposition.png}
      \caption{Manually segmentation of WalkingForward}
  	\end{minipage}
	\end{figure}
	
\end{frame}

\begin{frame}
	\frametitle{Template Construction}
	
	\begin{alertblock}{Goal}

		Build a model of the elementary movement of our activities. Then, one will compare the template and a sub-segment of the time series to be recognized.

	\end{alertblock}
	
	\begin{block}{How to construct the template}
	There is two way to construct the template : 
	\begin{itemize}
		\item averaging the segments
		\item using DBA method developped by Petitjean\cite{petitjean} which average under time wraping.
	\end{itemize}	  
	\end{block}
	
\end{frame}

\begin{frame}
	\frametitle{Template Construction}
	\framesubtitle{The averaging method : an awful solution}
	
	On the following picture, one represents in blue the average segment and in red the associated variance. The variance is very hight ! This is due to the abscissa shift between the segment.
	
	\begin{figure}
	\includegraphics[scale=0.4, center]{../report_pictures/segment_moyen.png}
	\caption{Building the template by averaging the segments}
	\end{figure}
	
\end{frame}

\begin{frame}
	\frametitle{Template Construction}
	\framesubtitle{The averaging method : an awful solution}
	
	\begin{alertblock}{As a result :}
	This method is not exploitable !
	\end{alertblock}

\end{frame}

\begin{frame}
	\frametitle{Template Construction}
	\framesubtitle{The DBA method}
	
	\begin{alertblock}{Goal}
		Create average centroids that are consistent with the warping behavior of DTW. 
	\end{alertblock}
	
	DBA iteratively refines an average sequence $a$ and follows an expectation-maximization scheme:
	\begin{enumerate}
		\item Consider the average sequence $a$ fixed and find the best multiple alignment $M$ of the set of sequences $S$ with regard to $a$, by individually aligning each sequence of $S$ to $a$.
		\item Now consider $M$ fixed and update $a$ as the best average sequence consistent with $M$.
	\end{enumerate}
\end{frame}

\begin{frame}
	\frametitle{Template Construction}
	\framesubtitle{DBA Results}
	
	\begin{figure}[H]
		\centering
		\includegraphics[scale=0.4,center]{../USC-Activities/RunningForward/manual/breaking_points.png}
		\caption{The manaul segmentation of RunningForward}
		\label{bp_RF}
	\end{figure}
	
\end{frame}

\begin{frame}
	\frametitle{Template Construction}
	\framesubtitle{DBA Results}
	
	\begin{figure}[H]
		\centering
		\includegraphics[scale=0.4,center]{../USC-Activities/RunningForward/manual/segments_superposition.png}
		\caption{Segments superposition for RunningForward}
		\label{seg_sup_RF}
	\end{figure}
   
\end{frame}

\begin{frame}
	\frametitle{Template Construction}
	\framesubtitle{DBA Results}
	
	\begin{figure}[H]
		\centering
		\includegraphics[scale=0.55,center]{../data/AverageSegments/RunningForward.png}
		\caption{Average segment for RunningForward}
		\label{av_seg_RF}
	\end{figure}
   
\end{frame}

\begin{frame}
	\frametitle{Template Construction}
	\framesubtitle{The weight of the medoid}
	
	\begin{exampleblock}{Notes}
	The variance has no longer any sense with the DBA method, so one has to compute an equivalent : the weight of the centroid. We use the idea developped by Goncharov\cite{goncharov} to compute the weight vector. The weight was in red on the previous picture.
	\end{exampleblock}
	
\end{frame}


\begin{frame}
	\frametitle{Time Series Comparaison}
	\framesubtitle{Presentation of the model}
	
	\begin{alertblock}{Justification}
	To recognize and classify the time series, one has to be able to compare them. A time serie must be compared to all the patterns at the same time so as to detect to which pattern it most looks like.
	\end{alertblock}
	
	So, one has to first compare a segment and a template and to characterize how near they are.
	
	\begin{exampleblock}
	It's a hard problem, and it's the core of our work. The problem have not at the present time been entirely solved.
	\end{exampleblock}
	
\end{frame}

\begin{frame}
	\frametitle{Time Series Comparaison}
	\framesubtitle{Caracterize the distance between a template $c$ and a segment $s$}

	Let $c=\{c_i\}_{ i \in \{1, ..., n\} }$ a series template of length $n$ and $s=\{s_i\}_{ i \in \{1, ..., m\} }$ a time series of length $m>n$.

	
	\begin{alertblock}{Problem}
	Spatio-temporel shift.
	\end{alertblock}
	
	\begin{block}{Modelisation}
	$c(t) = w_1 s(w_3 t + w_2) + w_0$
	\end{block}	
	
	\begin{exampleblock}{Notes}
	\begin{itemize}
	 \item $w0$ is the \textit{floor} of the segment compared to the pattern
	 \item $w1$ is the amplitude ratio
	 \item $w2$ is the advance ( $ >0$ ) or the delay ($<0$)
	 \item $w3$ is the speed ration
	\end{itemize}
	\end{exampleblock}
	
\end{frame}

\begin{frame}
	\frametitle{Time Series Comparaison}
	\framesubtitle{Compute the temporal-shift parameter}
	
	Via DTW, we compute the optimal path and the associated weight between the template $c$ and the segment $s$ :
	
	\begin{figure}[H]
		\centering
		\includegraphics[scale=0.4,center]{../report_pictures/exemple_chemin.png}
		\caption{An exemple of path}
		\label{ex_dist_rep}
	\end{figure}
	
\end{frame}

\begin{frame}
	\frametitle{Time Series Comparaison}
	
	We denote the optimal path by : $\Pi = \{(p_{s_i}, p_{c_j})\}_{i,j \in \{1, ..., n\} \times \{1, ..., m\} }$
	
	The path has a length $p$.
	
	The weight of the optimal path: $R = \{ \rho_i \}_{i \in \{1, ..., p\} } = \{ (c_{k_i} - s_{k_j})^2 \}_{k \in \{1, ..., p\} }$ with $(k_i,k_j)$ the k-eme couples of $\Pi$
	
	\begin{alertblock}{Goal}
		Get a linear regression : $d = a \times x + b$ of this path so as to compute $w_2$ (the delay) and $w_3$ (the speed).
	\end{alertblock}	
	
	\begin{block}{$w_2$ and $w_3$}
	\begin{itemize}
		\item $w_3=a$
		\item $w_3 \times (-w_2) + b = 0 \Leftrightarrow w_2 = \frac{a}{b}$
	\end{itemize}
	\end{block}
	
\end{frame}

\begin{frame}
	\frametitle{Time Series Comparaison}
	\framesubtitle{The influence of $w_2$}

	\begin{figure}
   	\begin{minipage}[c]{.46\linewidth}
      \includegraphics[scale=0.3]{../report_pictures/retard.png}
      \caption{$w_2 < 0$ : a delay}
   	\end{minipage} \hfill
   	\begin{minipage}[c]{.46\linewidth}
      \includegraphics[scale=0.3]{../report_pictures/avance.png}
      \caption{$w_2 > 0$ : an advance}
  	\end{minipage}
	\end{figure}

\end{frame}

\begin{frame}
	\frametitle{Time Series Comparaison}
	\framesubtitle{$w2$ and $w3$}
	
	\begin{figure}[H]
		\centering
		\includegraphics[scale=0.3,center]{../report_pictures/path.png}
		\caption{The path matrix}
		\label{ex_dist_rep}
	\end{figure}
	
\end{frame}


\begin{frame}
	\frametitle{Time Series Comparaison}
	\framesubtitle{$w2$ and $w3$}
	
	\begin{figure}[H]
		\centering
		\includegraphics[scale=0.3,center]{../report_pictures/regression.png}
		\caption{The regression of the path, here $w2=39$ and $w3=0.9938$. This is correct}
		\label{ex_dist_rep}
	\end{figure}

\end{frame}

\begin{frame}
	\frametitle{Time Series Comparaison}
	\framesubtitle{The bound of $w2$ and $w3$}
	
	The temporal-shift parameters must verify some bounds constraints.
	
	\begin{itemize}
	\item $w2 \in [0, m-n]$ where $m$ is the length of the segment and $n$ the length of the template	
	\item $w3 \in [\Un_{-}, \Un_{+}]$
	

	\end{itemize}
	
	The parameters must be computed under this constraints. Yet the constraints are not verify in the general case ! 
	
\end{frame}

\begin{frame}
	\frametitle{Time Series Comparaison}
	\framesubtitle{Compute the spatial-shift}
	
	Without considering the temporal-shift it's quite a simple problem. Indeed it's a least-square method.
	
	\[ d = \underset{w_0,w_1}{min} (t - w_1 s - w_0)^T B (t - w_1 s - w_0) = \underset{w_0,w_1}{min} \sum \frac{(t_i - w_1 x_i - w_0)^2}{\sigma_i^2} \]
	
	At the present time this solution has been implemented correctly, and the following results have been obtain with this solution.
	
	\begin{exampleblock}{Improvements}
	
	When the temporal parameter would have been computed, the problem will become more tricky. Indeed, one would have to reel in the segment to the template.
	\end{exampleblock}
	
\end{frame}

\begin{frame}
	\frametitle{Time Series Comparaison}
	\framesubtitle{Remaining question}

	\begin{itemize}
		\item The bounds of $w_i, i \in {1,2,3,4}$
		\item The interpolation of $s$ to reel in the segment to the template
	\end{itemize}		
\end{frame}

\begin{frame}
	\frametitle{Time Series Classification}

	We define the following distance vector for a serie regarding to all the template :
	
	\[ d=[w_{0,1}, w_{1,1}, w_{2,1}, w_{3,1}, w_{0,2}, ..., w_{3,12}] \in \R^{48} \]
	
	With this distance vector we obtain the following data distributions. 
	
	\begin{figure}[H]
		\centering
		\includegraphics[scale=0.22]{../data/DistanceVectorComponents/3_11.png}
		\caption{Components $3$ and $11$ of the distance vector ie $w_{1,2}$ and $w_{1,6}$}
		\label{dist_components}
	\end{figure}
	
\end{frame}

\begin{frame}
	\frametitle{Why we do not compute a distance in $\R$}
	
	\begin{figure}
      \includegraphics[scale=0.25,center]{../data/ImagesOfRecognition/WalkingForward_REC_WalkingForward.png}
      \caption{WalkingForward recognizing WalkingForward}
   	\end{figure} 
   	\begin{figure}
      \includegraphics[scale=0.25,center]{../data/ImagesOfRecognition/WalkingForward_REC_ElevatorDown.png}
      \caption{WalkingForward recognizing ElevatorDown}
	\end{figure}
	
\end{frame}

\begin{frame}
	\frametitle{Test and Performance}
	\framesubtitle{Test of the recognition}
	
	With the half-computed solution (considering only the spatial shift) we have measured the accuracy of the recognition using a cross-vaidation and a hold-out-validation for four different classifiers :
	
	\begin{itemize}
		\item Support Vector Machine
		\item Gaussain Naive Bayesian
		\item k-Nearest Neighbors
		\item Decision Tree
	\end{itemize}
		
\end{frame}

\begin{frame}
	\frametitle{Test and Performance}
	\framesubtitle{Test of the recognition}
	
	\begin{exampleblock}{How to read the following graphics}
	The vertical axis corresponds to the activities one wants to recognize. The horizontal axis corresponds to the activities recognized by the algorithm.
	
	For instance the value at the coordinate $(1,2)$ match with the percentage of serie recognized as the second activity while in reality it is the first activity. 
	
	Thus, the detection would be perfect if for every data set, one would obtain a diagonal matrix.
	\end{exampleblock}
\end{frame}

\begin{frame}
	\frametitle{Test and Performance}
	\framesubtitle{Test of the recognition : SVM}
			
	\begin{figure}[H]
			\begin{minipage}[c]{.46\linewidth}
      			\includegraphics[scale=0.42,center]{../report_pictures/auto_recognition_svm.png}
  			\end{minipage} \hfill
   			\begin{minipage}[c]{.46\linewidth}
      			\includegraphics[scale=0.42,center]{../report_pictures/recognition_svm.png}
   			\end{minipage}
		\caption{Auto and Hold-Out Validation for SVM}
		\label{svm}
	\end{figure}
	
\end{frame}

\begin{frame}
	\frametitle{Test and Performance}
	\framesubtitle{Test of the recognition : Gaussian Naive Bayes}
			
	\begin{figure}[H]
			\begin{minipage}[c]{.46\linewidth}
      			\includegraphics[scale=0.42,center]{../report_pictures/auto_recognition_gnb.png}
  			\end{minipage} \hfill
   			\begin{minipage}[c]{.46\linewidth}
      			\includegraphics[scale=0.42,center]{../report_pictures/recognition_gnb.png}
   			\end{minipage}
		\caption{Auto and Hold-Out Validation for Gaussian Naive Bayes}
		\label{gnb}
	\end{figure}
	
\end{frame}

\begin{frame}
	\frametitle{Test and Performance}
	\framesubtitle{Test of the recognition : k-nearest neigbors}
			
	\begin{figure}[H]
			\begin{minipage}[c]{.46\linewidth}
      			\includegraphics[scale=0.42,center]{../report_pictures/auto_recognition_knn.png}
  			\end{minipage} \hfill
   			\begin{minipage}[c]{.46\linewidth}
      			\includegraphics[scale=0.42,center]{../report_pictures/recognition_knn.png}
   			\end{minipage}
		\caption{Auto and Hold-Out Validation for k-nearest neigbors}
		\label{knn}
	\end{figure}
	
\end{frame}

\begin{frame}
	\frametitle{Test and Performance}
	\framesubtitle{Test of the recognition : Decision Tree}
			
	\begin{figure}[H]
			\begin{minipage}[c]{.46\linewidth}
      			\includegraphics[scale=0.42,center]{../report_pictures/auto_recognition_dtc.png}
  			\end{minipage} \hfill
   			\begin{minipage}[c]{.46\linewidth}
      			\includegraphics[scale=0.42,center]{../report_pictures/recognition_dtc.png}
   			\end{minipage}
		\caption{Auto and Hold-Out Validation for Decision Tree}
		\label{dtc}
	\end{figure}
	
\end{frame}

\begin{frame}
	\frametitle{TODO : What I have to do}
	
	\begin{itemize}
	\item Compute without error $w2$ and $w3$
	\item Carry out the temporal shift and compute $w0$ and $w2$.
	\item Implement the fast dtw algorithm
	\item Think how to speed up the comparaison
	\item Introduce in the classification a \textit{draft} class
	\item Take into count the continues aspect of the time
	\item Think about the exceptionally long time of the escalator pattern with  regards to the rest.
	\item Write all the test (incomplet at the present time)
	\item Write all the document (incomplet at the present time)
	\end{itemize}
\end{frame}	

%\begin{frame}[allowframebreaks]
%        \frametitle{References}
%        \bibliographystyle{bibidx/fr-plain}
%        \bibliography{biblio.bib}
%\end{frame}



\end{document}